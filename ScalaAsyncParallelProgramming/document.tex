\documentclass[12pt]{article}

\usepackage{url} % websites in bib
\usepackage{hyperref}
\usepackage[a4paper,left=1in,right=1in,top=1in,bottom=1in]{geometry} % margins

\usepackage{amsmath}
\usepackage{amssymb}
\usepackage{commath}

\usepackage{todonotes}
\usepackage{float} % figure placement
\usepackage[]{algorithm2e}

\usepackage{listings} % for code


\definecolor{dkgreen}{rgb}{0,0.6,0}
\definecolor{gray}{rgb}{0.5,0.5,0.5}
\definecolor{mauve}{rgb}{0.58,0,0.82}

\lstdefinestyle{myScalastyle}{
  frame=tb,
  language=scala,
  aboveskip=3mm,
  belowskip=3mm,
  showstringspaces=false,
  columns=flexible,
  basicstyle={\small\ttfamily},
  numbers=none,
  numberstyle=\tiny\color{gray},
  keywordstyle=\color{blue},
  commentstyle=\color{dkgreen},
  stringstyle=\color{mauve},
  frame=single,
  breaklines=true,
  breakatwhitespace=true,
  tabsize=3,
}

\begin{document}
	\title{Scala, asynchronous parallel programming}
	\author{Stefan Sebastian, 242}
	\date{}
	\maketitle
	
	\newpage
	\tableofcontents
	\newpage

	\section{Introduction}
	
	\subsection{The Scala programming language}
	Scala is a programming language designed to integrate smoothly both object-oriented and 
	functional programming created with the purpose 
	of building scalable components and component systems\cite{AnOverviewOfTheScalaProgrammingLanguage}. 
	Scala was designed to work well with some of the most popular programming languages: Java and Javascript. 
	The syntax is very similar to that of Java, and by extension C\#, making transition 
	easier for programmers. Furthermore, Scala interoperates well with the JVM and can use most of the features 
	of Java, including its rich ecosystem comprised of thousands of already existing 
	and tested libraries. It is worth noting that it was designed to address most criticisms of Java.
	As of 2013, Scala can also be transpiled into highly performant 
	Javascript code and run in the browser\cite{ScalaJs} and also make use of popular frameworks such as
	React and Angular.

	Scala is a functional language, because it treats functions as values, supports nested 
	functions and provides a concise syntax for anonymous and curried functions\cite{AnOverviewOfTheScalaProgrammingLanguage}.
	It is also a pure object oriented language in the sense that every values is an object, including primitives and functions. 
	Some other notable features include: pattern matching, static typing, compiler inference of most types of variables.

	\subsection{Scala parallel and concurrent programming}

	Scala allows the programmer to use all parallel and concurrent constructs of the host language, for example 
	Java Threads or the utility classes from the java.util.concurrent package, and also adds some specific tools.
	The main additions are its actor library and extension of the Future and Promise concepts. These additions 
	will be the focus of this paper.

	\section{Futures and Promises}

	\subsection{Future}
	The Future is one of Scala's approaches in dealing with concurrency\cite{FuturePromiseScalaDocs}. 
	The main motivation for it is to provide a concise way of writting asynchronous code, mainly 
	to avoid blocking application threads during IO operations. 

	A Future[T] is a container type, meaning that it will eventually contain a value of type T, or an exception\cite{FuturePromiseNeophyte}.
	It encapsulates a computation, which, upon completion, will result in the respective value. A Future 
	can only be written once, therefore its value can not be changed after the assignment caused by the 
	end of the computation, or exception flow.

	\begin{lstlisting}[style=myScalastyle]
		object Future {
			def apply[T](body: => T)(implicit execctx: ExecutionContext): Future[T]
		}
	\end{lstlisting}

	One way to create a Future is through the apply method on the Future companion object.
	A companion object is a singleton object with the same name as a class and contains methods 
	and values specific to all instances of the companion class\cite{SingletonObj}. The apply method
	takes to parameters: a computation which returns a value of type T, and which will be executed 
	asynchronously, and an ExecutionContext, which is an implicit parameter, meaning that it is optional 
	if defined somewhere in scope. 

	An ExecutionContext is responsible for executing computations and is very similar to the 
	Java concept of Executor\cite{FuturePromiseScalaDocs}. It can execute operations in the same thread, a new thread or a 
	pooled thread. The scala.concurrent package comes with a default implementation, ExecutionContext.global, 
	which is backed by a ForkJoinPool, from the Java concurrent package. 
	The number of threads managed by it is called the parallelism level, which is set 
	by default to the number of available processors. If needed a Java Executor can also be adapted 
	using the method ExecutionContext.fromExecutor(...).

	\begin{lstlisting}[style=myScalastyle, caption={Methods for creating Future objects}, label={futureDef}]
		type CoffeeBeans = String
		type GroundCoffee = String
		case class Water(temperature: Int)

		def heatWater(water: Water): Future[Water] = Future {
			println("started heating water")
			Thread.sleep(Random.nextInt(2000))
			println("water is hot!")
			water.copy(temperature = 85)
		}

		def grind(beans: CoffeeBeans): Future[GroundCoffee] = Future {
			println("start grinding")
			Thread.sleep(Random.nextInt(2000))
			println("finished grinding")
			s"ground coffee of $beans"
		}
	\end{lstlisting}

	In figure \ref{futureDef} there is an example of two functions which create Future objects using 
	the Future companion object, taken from the Neophyte guide to Scala\cite{FuturePromiseNeophyte}. 
	The apply method is defined inside the braces and the execution context is assumed to be implicit.
	After a Future instance is created, the ExecutionContext will assign a thread to it, on which 
	the computation will be executed at a non-determinstic time.

	In order to capture the result and do something with it, a callback can be added to a Future 
	object, as demonstrated in figure \ref{callback}. The callback can handle both success and 
	exception cases. 
	\begin{lstlisting}[style=myScalastyle, caption={Registering callbacks}, label={callback}]
		import scala.util.{Success, Failure}
		grind("arabica beans").onComplete {
			case Success(ground) => println(s"got my $ground")
			case Failure(ex) => println("could not grind")
		}
	\end{lstlisting}

	Using callbacks can quickly lead into what is referred to as 'callback hell'\cite{FuturePromiseNeophyte}
	which means having a lot of nested functions which are difficult to read. However, Scala provides 
	a useful alternative, which is mapping futures. 

	For example, in case we want to check if after heating up the water its temperature is ok we 
	can map a Future[Water] to a Future[Boolean] like in figure \ref{mapFuture}. The tempOk value 
	will be assigned a Future[Boolean]. The mapping function gets executed as soon as the 
	Future[Water] has completed successfully, however it is executed synchronously.
	\begin{lstlisting}[style=myScalastyle, caption={Mapping future}, label={mapFuture}]
		val tempOk: Future[Boolean] = heatWater(Water(25)).map { water =>
			(80 to 85).contains(water.temperature)
		}
	\end{lstlisting}

	In the case in which we want the mapped function to also be computed asynchronously we need 
	to keep in mind that the result will be a Future[Future[T]], which is not easy to work with. 
	For this reason, Scala provides the flatMap utility which composes futures and returns a Future[T].

	\begin{lstlisting}[style=myScalastyle, caption={FlatMapping future}, label={flatMapFuture}]
		def tempOk(water: Water): Future[Boolean] = Future {
			(80 to 85).contains(water.temperature)
		}
		val flatFuture: Future[Boolean] = heatWater(Water(25)).flatMap {
			water => temperatureOkay(water)
		}
	\end{lstlisting}

	Scala also provides a shorter way to compose multiple futures: the for comprehension 
	\cite{FuturePromiseNeophyte}.
	It should be pointed out that the futures are started before the for block so they are executed 
	concurrently, while the brew method will wait until ground and water values become available.

	\begin{lstlisting}[style=myScalastyle, caption={For comprehensions}, label={forComp}]
		def prepareCoffee(): Future[Coffee] = {
			val groundBeans = grind("arabica beans")
			val heatedWater = heatWater(Water(20))
			for {
				ground <- groundBeans
				water <- heatedWater
				coffee <- brew(ground, water)
			} yield coffee
		}
	\end{lstlisting}

	\subsection{Promise}
	Promise is a companion type for Future that allows you to complete it by setting a value\cite{FuturePromiseNeophyte}.
	This operation can only be done once. A Promise is always linked with exactly one Future instance.
	Even the apply method from the Future companion object creates a Promise, but shields the user 
	from doing so explicitly.

	\begin{lstlisting}[style=myScalastyle, caption={Promise example}, label={promise}]
		val promise: Promise[String] = Promise[String]()
		val future = promise.future 
		...
		val anotherFuture = Future {
			...
			promise.success("Done")
			doSomethingElse()
		}
		...
		future.onSuccess { case msg => startTheNextStep() }
	\end{lstlisting}

	The basic operations in working with Promises is illustrated in figure \ref{promise}, taken 
	from the Scala in Action book\cite{ScalaInAction}. There are two Futures being created, one 
	using the Promise object explicitly and one using the apply method. Inside the second 
	Future the success method is invoked which will cause the onSuccess callback registered 
	on the first Future to be executed.

	\subsection{Comparison with other languages}
	The Promise and Future approach to working with asynchronous code is available 
	in most of the other popular programming languages, however the Scala version 
	offers some improvements. 

	First of all, the Java Future class provides a very limited interface\cite{FutureJava}. For example, 
	it can not register a callback which results in messy code that has to do periodic polling 
	for the result. An alternative from the java.util.concurrent package is the FutureTask, which 
	allows the programmer to register callbacks. However, it does not provide a way to register 
	a callback for exceptions, meaning that aspect is up to the programmer. Some ways to do that are 
	to subclass a FutureTask and override the done() method or to surround the get() method from 
	the Future object with a try-catch block, both approaches being much more verbose than the Scala 
	solution.

	A language which makes heavy use of this concept is Javascript, especially Node.js, which being 
	single threaded needs to perform operations in a non-blocking way\cite{FutureJs}. The Future objects from 
	JS are very similar to those in Scala. An advantage of the Scala Future approach is that it 
	provides a concise way of chaining multiple Futures and setting their dependencies using 
	for-comprehensions, like in figure \ref{forComp}.

	\subsection{In practice}
	The main use of the future-based programming paradigm is to create scalable web applications\cite{FuturePromiseNeophyte}.
	The issue with traditional web servers are times spent waiting for IO operations like database 
	queries or while interacting with other services, which causes the execution thread to block 
	waiting for a response. Using a modern Scala web framework you can return the responses as Future[Response]
	instead of blocking waiting for the results. This means that all layers of the application must be 
	asynchronous and the result should be a composition of all these Futures. This will free up 
	the server to handle more requests and increase the scalability of the application.

	\begin{lstlisting}[style=myScalastyle, caption={Wrapping blocking operations with Futures to increase scalability}, label={wrapBlocking}]
		Future {
			queryDB(query)
		}
	\end{lstlisting}

	The cases when a Future should be used over the actor model are when performing an 
	operation concurrently and the extra utility of the actor is not necessary, or when 
	there is a need to compose multiple concurrent operations, which is difficult to do 
	with actors\cite{ScalaInAction}. 

	\section{The Actor model}
	\subsection{Model overview}
	The actor model is a computational model designed to deal with concurrency at a high 
	level of abstraction\cite{ProgrammingInScala}. It represents a specific implementation 
	of the message passing concurrency model, which encourages shared nothing architectures.
	The building block of the model is the Actor, 
	which is the primitive unit of computation that receives a message and handles it.
	This is similar to how objects work in the object-oriented paradigm, however actors 
	never share memory, which eliminates a lot of the issues of concurrent programming.

	Actors always process messages sequentially, which can be sent both synchronously or 
	asynchronously, with the latter being more common\cite{ScalaInAction}. While the actor 
	processes a message, all the other incoming ones are stored in a queue, one for each 
	actor, called "the mailbox". The messages themselves are immutable and are the only 
	way to cause actors to change their internal state.

	\subsection{The Actor model in Scala}
	Scala provides an implementation for the message passing concurrency model through its 
	actor library\cite{ScalaInAction}. After Scala 2.10 the default actor library is Akka, which is also 
	available for Java. There are other libraries but this is by far the most popular. 

	Actors have two main communication abstractions: send an receive. The syntax for 
	sending a message is very simple: actor ! message. The 'message' is then added to
	the 'actor' mailbox which is a first-in first-out queue. In order to receive a message 
	each actor has to define a function called 'receive' which is usually a set of patterns 
	matching message types to specific actions.
	
	\begin{lstlisting}[style=myScalastyle, caption={Simple actor implementation}, label={greetingActor}]
		import akka.actor.Actor 
		case class Name(name: String)
		class GreetingsActor extends Actor {
			def receive = {
				case Name(n) => println("Hello " + n) 
			}
		}
	\end{lstlisting}

	You can reply to messages using the sender method of an ActorRef\cite{AkkaDocs} like this: sender() ! x.
	There is a pattern of mixing future and actors if we need to call the reply method when 
	some asynchronous task completes and we don't want to block the executing thread.

	\subsubsection{Actor Systems}
	The infrastructure which is required for running actors is created by an 
	actor system\cite{ScalaInAction}.
	An actor system is a manager of one or more actors which share a common configuration. An actor 
	can be instantiated through the system using the method actorOf which takes a configuration object 
	and, optionally, the actor's name. The actor system can be destroyed using the 'shutdown' method,
	which also stops all actors, even those who still have unprocessed messages. 

	\begin{lstlisting}[style=myScalastyle, caption={Actor system demo}, label={actorSystem}]
		import akka.actor.Actor
		import akka.actor.ActorSystem 
		val system = ActorSystem("greetings")
		val a = system.actorOf(Props[GreetingsActor], name="greetings-actor")
	\end{lstlisting}

	Actor systems are heavyweight structures, so ideally there should be one per application 
	module: one for the database layer, one for web service calls, etc\cite{AkkaDocs}. An actor system maintains 
	a hierarchy of all actors associated with it. At the top, there is the guardian actor, which is 
	created automatically. Each actor in a system has a parent, which is responsible for error 
	handling, and can have multiple children.

	\subsubsection{Actor references}
	When you create actors in Akka you get back a reference called ActorRef to the actual 
	object\cite{ScalaInAction}. This means that you can't access the actor directly and mutate its state. The ActorRef
	is also serializable, so if an actor crashes in one node of a distributed system, you 
	can send the reference to another node and start it there. The actor system also mainatins an actor 
	path, which is an unique identifier for each actor, given the fact that actors are maintained 
	in a hierarchical structure. A useful thing about the actor path is that it can also identify 
	actors on other machines.

	\subsubsection{Message dispatchers}
	The actual job of sending and receiving messages in an actor system is done by the MessageDispatcher\cite{ScalaInAction} 
	component. Every actor system comes with a default one, however this can be overriden through configuration
	files and can be specialized for different actors. Every MessageDispatcher uses a thread pool internally.
	When receiving a message, the dispatcher adds it to the actor mailbox and immediately returns control 
	to the sender. Sending and handling messages happens in different threads. When an actor receives 
	a message the dispatcher checks if there is a free thread in the pool, and in case there isn't waits 
	for one to be available. The selected thread reads messages from the mailbox and executes them 
	by passing the receive method sequentially. A guarantee offered by the dispatcher is that a single 
	thread always executes a given actor, which means that you can safely use mutable state inside an 
	actor. 

	
	\begin{lstlisting}[style=myScalastyle, caption={Actor API}, label={actorAPI}]
		def unhandled(message: Any): Unit // for unmatched messages 
		val self: ActorRef // reference to self
		final def sender: ActorRef // reference to sender of last received message 
		val context: ActorContext // contextual information for the actor 
		def supervisorStrategy: SupervisorStrategy // fault handling strategy 
		def preStart() // when started for the first time 
		def preRestart() // when restarted 
		def postStop() // after being stopped 
		def postRestart() // after restart 
	\end{lstlisting}

	\subsubsection{Supervision hierarchy}
	Each actor is the supervisor of its children and has associated with it a 
	SupervisorStrategy\cite{AkkaDocs}. Akka encourages non-defensive programming, 
	which is a paradigm that treats errors as a valid state of the application 
	and handles it through fault tolerance methods, such as the supervisor strategies. 
	Akka provides two supervisor strategies: One-for-One and All-for-One. The first 
	one means restarting only the actor that dies, which is useful if actors are independent 
	in the system. However, in some cases if the workflow is dependent on multiple actors 
	interacting then restarting only one might not work. For this reason the second strategy 
	was introduced, which restarts all the actors supervised by the supervisor. 

	The responsability of a supervisor is to start, stop and monitor the child actors\cite{ScalaInAction}. 
	Each actor is by default a supervisor for its child actors. You can also have the supervisor 
	hierarchy spread across multiple machines, which greatly increases the fault tolerance 
	of the system.

	\subsection{Advantages and disadvantages of the Actor Model}
	The actor model was designed as a high level approach for concurrent programming,
	and it solves most of the problems associated with the classic Java Thread based approach\cite{AkkaDocs}.
	One of the areas which causes most bugs and performance hits in the thread model is shared mutable 
	state. This means that multiple threads can access and modify the same variables which can 
	lead to inconsistent state and requires synchronization mechanisms such as locks.
	Actors encapsulate state and the only way to modify the internal state of an actor 
	is through the messages it receive. Furthermore, actors process messages in a synchronous 
	fashion, meaning that the same data is never accessed concurrently. This approach already 
	solves a lot of potential concurrency related bugs. 
	
	Another problems with synchronization 
	is the necessity for locks. Locks are very costly and limit concurrency. Also, the 
	caller thread is blocked, meaning it won't be able to do any meaningful work. This could be 
	compensated by launching more threads, but threads are a costly abstraction and add significant 
	overhead. Moreover, programming with locks requires a lot of careful consideration from the 
	part of the programmer, because it can introduce a whole new class of bugs: deadlocks.
	
	Most of the shared memory models were designed back when writting to a variable meant 
	writting directly to a memory location. However, modern architectures have multiple CPU's 
	and writes are done to cache lines. This means that in order to make changes visible 
	between different cores they need to be shipped explicitly to the other core's cache, 
	which causes a drop in performance.

	Another issue with the above model is error handling\cite{AkkaDocs}. Call stacks are local to a thread 
	so when a thread has to delegate work to another and the callee fails we can't rely 
	on the call stack. Actors are designed to handle errors more gracefully. Firstly if 
	the delegated task fails because of an error in the task than the erorr is part of the 
	problem domain and the response presents the error case. Secondly, if the service itself 
	encounters an internal fault, its parent actor is notified and has the responsability
	to restart the children, just like processes are organised in operating systems. The main 
	idea is that children never die silently.

	However, the actor model is not always the best tool for the task. A classic example 
	of where this wouldn't be a good fit is for shared state\cite{ScalaInAction}. If you want to transfer 
	money from one account to another you can't do that only through actors, you need 
	transactional support. Another issue is that switching to asynchronous programming 
	introduces a time cost for the developers. It is a huge paradigm shift and message 
	passing architectures make for hard to debug programs. Furthermore, there are some 
	cases when you need maximum performance. In such situations using actors adds overhead 
	and a more suitable solution is going for something low level like threads.

	\subsection{Comparison Erlang Actor Model implementations}
	One of the most popular actor model implementation is the one provided by Erlang, 
	which is a language designed to be parallel and distributed with roots in the 
	telecom industry, where concurrent processes are the norm.

	Erlang has better fault tolerance than scala because its VM forces full isolation 
	between actors, even having separate heaps and garbage collection\cite{ErlangVsScala}. This is something 
	that can't be done in Scala since it is built on the JVM.

	Erlang does the process scheduling for you, using a preemptive scheduler\cite{ErlangVsScala}. This means 
	that when a process has been executed for too much time it is paused and another 
	actor is started. However, you can not change the default scheduler, something which 
	is highly customizable in Scala (using Akka).

	The platform on which Erlang runs is OTP(Open Telecom Platform) which allows for hot swapping of code\cite{ErlangVsScala}. 
	This means that a real time system can be upgraded without having downtime. In Scala this is not 
	possible due to JVM limitations. Because of the static type system method signature can not 
	be changed, only their content. However, an advantage of running on the JVM is that Akka can 
	be integrated with the entire Java ecosystem, which has a tool for almost every task.

	\subsection{An example of an Actor based application}

	An example of a more complex application written using the actor model can be seen 
	in figure\ref{wordcount}, which was taken from the Scala in Action book\cite{ScalaInAction}.
	The problem statement is to count the number of words in each file of a directory and 
	then sort them in ascending order. The approach taken is divide et impera, with a master actor 
	that creates a number of slaves charged with reading a file and counting the words. The master 
	then combines and sorts the results. 

	The worker actor is conceptually simpler. The receive message can handle just one type 
	of message which contains a file that the actor should read and process. After counting 
	the words the worker sends a message to sender containing a touple of file and number of 
	words.

	The master actor can handle two types of messages. First of all, the StartCounting message 
	is sent at the beggining of the workflow, to initialize the process. Depending on the given 
	parameters, the master will create child actors and read files from the required location. 
	Because there may be more files than actors, the FileToCount message is sent to each 
	actor in a round robin fashion. The reply from each worker is a WordCount message, which 
	is used to update the sortedCount map, which maintains the file names alongside their 
	word count. When all files have been counted the actor system is terminated and the application 
	stops.

	\begin{lstlisting}[style=myScalastyle, caption={Actor wordcount}, label={wordcount}]
	import akka.actor.Actor
	import akka.actor.Props
	import akka.actor.ActorRef
	import akka.actor.ActorSystem
	import java.io._
	import scala.io._

	case class FileToCount(fileName:String)
	case class WordCount(fileName:String, count: Int)
	case class StartCounting(docRoot: String, numActors: Int)

	class WordCountWorker extends Actor {
		def countWords(fileName:String) = {
			val dataFile = new File(fileName)
			Source.fromFile(dataFile).getLines.foldRight(0)(_.split(" ").size + _)
		}
		def receive = {
			case FileToCount(fileName:String) =>
				val count = countWords(fileName)
				sender ! WordCount(fileName, count)
		}
		override def postStop(): Unit = {
			println(s"Worker actor is stopped: ${self}")
		}
	}

	class WordCountMaster extends Actor {
		var fileNames: Seq[String] = Nil
		var sortedCount : Seq[(String, Int)] = Nil

		def receive = {
			case StartCounting(docRoot, numActors) =>
				val workers = createWorkers(numActors)
				fileNames = scanFiles(docRoot)
				beginSorting(fileNames, workers)
			case WordCount(fileName, count) =>
				sortedCount = sortedCount :+ (fileName, count)
				sortedCount = sortedCount.sortWith(_._2 < _._2)
				if(sortedCount.size == fileNames.size) {
					println("final result " + sortedCount)
					finishSorting()
				}
		}
		override def postStop(): Unit = {
			println(s"Master actor is stopped: ${self}")
		}
		
		private def createWorkers(numActors: Int) = {
			for (i <- 0 until numActors) yield context.actorOf(Props[WordCountWorker], name = s"worker-${i}")
		}
		
		private def scanFiles(docRoot: String) =
			new File(docRoot).list.map(docRoot + _)
		
		private[this] def beginSorting(fileNames: Seq[String], workers: Seq[ActorRef]) {
			fileNames.zipWithIndex.foreach( e => {
				workers(e._2 % workers.size) ! FileToCount(e._1)
			})
		}

		private[this] def finishSorting() {
			context.system.terminate()
		}
	}

	object Main {
		def main(args: Array[String]) {
			val system = ActorSystem("word-count-system")
			val m = system.actorOf(Props[WordCountMaster], name="master")
			m ! StartCounting("src/main/resources/", 2)
		}
	}

	\end{lstlisting}

	\section{Conclusions}
	Scala is a feature rich and flexibile language. Through its JVM implementation it can 
	access a very rich ecosystem and all parallel and asynchronous libraries available for Java. 
	On top of that, it comes with higher level constructs such as the Actor Model. The actor 
	model implementation offers many benefits over lower level thread based approaches in the 
	fact that it works as shared-nothing, it handles error gracefully and is very easy to scale 
	to multiple machines. Another approach for asynchronous code is the Future Promise model, 
	for which Scala provides a lot of syntactic sugar over other languages, making it easy 
	to work with.

	\newpage
	\bibliography{references}
	\bibliographystyle{ieeetr}
\end{document}